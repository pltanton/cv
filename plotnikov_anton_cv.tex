\documentclass{cv}

\usepackage{datetime}
\usepackage[super]{nth}
\newdateformat{monthyear}{\monthname[\THEMONTH], \THEYEAR}

\begin{document}

\cvheading{Anton Plotnikov}

\section{Personal information}

\entry{email}{\href{mailto:plotnikovanton@gmail.com}{plotnikovanton@gmail.com}}

\entry{telegram}{\href{https://t.me/pltanton}{@pltanton}}

\entry{location}{Saint Petersburg, Russia}

\entry{github}{\href{https://github.com/pltanton}{github.com/pltanton}}


\section{Key technologies}

\begin{cvblock}{I}
  Java/Kotlin, Ruby, Linux, Docker
\end{cvblock}

\begin{cvblock}{II}
  Go, Python, React.js, Haskell, \LaTeX, Nix (NixOS), JavaScript, TypeScript, k8s
\end{cvblock}

\begin{cvblock}{III}
  Bash, C++, C
\end{cvblock}


\section{Experience}

\begin{cvblock}{%
  \blocktitle{DSX Technologies}{St.~Petersburg}{Software engineer}{2018--present}}

  Development of cryptocurrency exchange/trade platform.

  \begin{itemize}
    \item \textbf{DSX}: Trade and exchange platform.
      \begin{itemize}
        \item Technologies: Kotlin, Java, Angular (administration site for support/aml team), Docker.
        \item Responsibilities: developement of platform backend, including:
          kyc/kyb system, exchange api, public UI api, authentication process,
          database and application design for clients personal data.
      \end{itemize}
    \item \textbf{Student projects}: Mentoring of student projects.
      \begin{itemize}
        \item Java subtyping system: project to provide solution to make additional
          subtyping checks for Java primitives in non-runtime via annotation processing.
        \item Rhea: library for dynamically and reactively reload configuration properties
          by changes in properties sources (file, database, vault, e.t.c). Based on
          Kotlin flow.
      \end{itemize}
    \item \textbf{DSXT Hackathon}: Participating in organisation and mentoring.
  \end{itemize}
\end{cvblock}

\vspace{2em}

\begin{cvblock}{%
  \blocktitle{Openway}{St.~Petersburg}{Software engineer}{2017--2018}}

  Developing infrastructure components for build registration
  system and workflow processes.

  \begin{itemize}
    \item \textbf{Releng}: Release registration system, which includes system itself
      (build registration, release/upgrade notes automatic generation,
      integration Jira plugin, clients to support CI by Teamcity).
      \begin{itemize}
        \item Technologies: Kotlin, Java, Python, Docker, Jira, Teamcity
        \item Responsibilities: support and developing system itself,
          building and planning architecture of some components.
      \end{itemize}
    \item \textbf{DMS}: Distribution management system. System to store and distribute
      business specific products builds with theirs release/upgrade notes.
      \begin{itemize}
        \item Technologies: Kotlin, Docker, Spring, React.js
        \item Responsibilities: support and developing system. Developing web
          UI service for access to client/release notes.
      \end{itemize}
  \end{itemize}
\end{cvblock}

\vspace{2em}

\begin{cvblock}{%
  \blocktitle{T-Systems RUS}{St.~Petersburg}{Software engineer}{2016--2017}}

  Developing web micro services based on Ruby on Rails framework and PL/SQL\@.
  Java development.

  \begin{itemize}
    \item \textbf{WiNA}:
      A system that stores operational network data on GSM,
      UMTS, etc.\ of T-Mobile Deutschland and the relevant data of
      network components. \textit{Technologies}: PL/SQL, CentOS, Oracle.
      \textit{Responsibilities}: improving error handling inside PL/SQL scripts,
      performing database optimizations

    \item \textbf{CenTr@}:
      An internal only-API service that gathers data on T-Systems RUS employees
      from different sources (SAP, corporate Active Directory,
      internal navigation system etc.), then processes and clears it out, and
      then provides flexible means for client systems to retrieve necessary
      data on demand. \textit{Technologies}: Ruby on Rails, MySql, Ubuntu.
      \textit{Responsibilities}: backend development, production server
      environment administration, deployment scripts administration,
      automatic Vagrant environment configuration

    \item \textbf{ExTr@}:
      An internal system that enables managerial personnel to track overtime
      and (extra-)events of T-Systems RUS employees and then build reports and
      employment forecasts.
      \textit{Technologies}: Ruby on Rails, PL/SQL, CentOS
      \textit{Responsibilities}: database planning, full-stack development,
      deployment configuration, database administration, automatic Vagrant
      environment configuration

\end{itemize}\end{cvblock}\begin{cvblock}{}\begin{itemize}

    \item \textbf{Calend@r}:
      An internal system that synchronizes a free-busy status from different
      Microsoft Exchange servers and builds a human readable comparison time
      table
      \textit{Technologies}: Sinatra, CoffeeScript, Ubuntu
      \textit{Responsibilities}: production server administration, deployment
      configuration, full-stack development, automatic Vagrant environment
      configuration
    \item \textbf{Gogs}:
      Administrating of Gogs local GIT service in CentOS environment. Related
      SSL, Nginx, smtp configuration and LDAP integration

  \end{itemize}
\end{cvblock}

\vspace{2em}

\begin{cvblock}{Open source}
  \begin{itemize}
    \item Network Widgets
      (\url{https://github.com/pltanton/net_widgets}):
                        a bunch of widgets, which indicates state of current network connection
      without NetworkManager dependency written in LUA for
      \textit{awesome} window manager
    \item YAGS
      (\url{https://github.com/pltanton/yags}):
      this program provides a simple configurable statusline generator. It
      passes the formatted status line each time when callbacks from plugging
      is received. Unlike conky, yags prints satatusline only when status
      really changed and not overloads disk with useless executions. Written in
      GO

  \end{itemize}
\end{cvblock}

\vspace{2em}

\begin{cvblock}{Freelance projects}
  \begin{itemize}
    \item Ruby on Rails course for Geekbrains:
      basic course of Ruby on Rails framework, designed for 8 lectures, which
      includes presentation for lecturer and book with detailed course
  \end{itemize}
\end{cvblock}

\section{Education}

\begin{cvblock}{%
  \blocktitle
    {ITMO University}
    {St.~Petersburg}
    {}
    {2016--2018}}
  Chair of Advanced Mathematics. Mathematical modeling.
  \vspace{1em}

  \textit{Master of Applied Math and Computer Science.}
\end{cvblock}

\begin{cvblock}{%
  \blocktitle
    {ITMO University}
    {St.~Petersburg}
    {}
    {2012--2016}}
  Chair of Advanced Mathematics. Mathematical modeling.
  \vspace{1em}

  \textit{Bachelor of Applied Math and Computer Science.}
\end{cvblock}

\section{Languages}

\begin{cvblock}{Russian}
  Native
\end{cvblock}

\begin{cvblock}{English}
  Intermediate
\end{cvblock}

\vfill
\begin{center}
  \monthyear\today
\end{center}

\end{document}
